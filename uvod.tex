\chapter{Uvod}

Danas najkorišteniji sustav za pozicioniranje, navigaciju i vremenske usluge je GPS \engl{Global Positioning System}. % s čijim razvojem je 1973. godine počelo Ministarstvo obrane Sjedinjenih Američkih Država.
U idealnim uvjetima preciznost određivanja lokacije GPS-a je oko deset metara, ali zbog toga što signali do uređaja dolaze od jako udaljenih satelita na nju utječe ogroman broj parametara, od zaklonjenog neba (npr. oblačno vrijeme pa čak i krošnje drveća) do metala u okolini uređaja. %TODO referencirati IEEE članak od Goluba
Dodatno, u zatvorenim prostorima dolazni signal satelita dodatno remete zidovi kroz koje signali prolaze i ostale prepreke u okolini.
Posljedično, korištenje GPS-a za pozicioniranje i navigaciju u zatvorenim prostorima je izrazito nepouzdano i nepraktično.
\\

Iz gore navedenih razloga očita je potreba za sustavom koji će moći pouzdano odrediti lokaciju korisnika u zatvorenom prostoru. 
Ovaj problem se našao u središtu velikog broja znanstvenih istraživanja pri čemu se većina tih istraživanja fokusira na određivanje lokacije na temelju signala bežične lokalne računalne veze \engl{wireless local area network , WLAN}. %, no prihvatljivih rješenja zasad nema. 
Uvođenjem Bluetooth 4.0 specifikacije i tehnologije \textit{Bluetooth Low Energy} dolazi do razvoja niza jeftinih uređaja male potrošnje koji se potencijalno mogu iskoristiti za rješavanje problema pozicioniranja i navigacije u zatvorenom prostoru.
\\

U ovom diplomskom radu govoriti će se o tehnologiji Bluetooth, problemu određivanja lokacije na zatvorenom prostoru, rješenju temeljenom na Bluetooth Smart odašiljačima, odnosno tehnologiji iBeacon, te smjernicama i savjetima za daljnji razvoj. 
U drugom dijelu rada prikazat ćemo kako se Bluetooth Smart odašiljači mogu primijeniti u mobilnim aplikacijama.