\chapter{Praktični rad}

U poglavlju o odašiljačima iBeacon spomenuto je kako se tehnologija iBeacon može iskoristiti u gotovo svim situacijama gdje se sadržaj mijenja ovisno o kontekstu prostora, npr. trgovinama, muzejima, bolnicama i sličnim objektima. 
\\
U nastavku poglavlja opisano je kako se tehnologija iBeacon može integrirati u trgovine. 
\\

Za integraciju je potrebno imati postavljene odašiljače iBeacon u prostoru, mobilni uređaj sa podrškom za BLE i instaliranom aplikacijom koja može prepoznati BLE, odnosno iBeacon, odašiljače te poslužiteljsku aplikaciju s koje će mobilni uređaj dohvaćati relevantne podatke.  

\section{Poslužiteljska aplikacija - Spree Commerce}
\label{sec:server}

Spree Commerce (u daljnjem tekstu Spree) je elektronička trgovina \engl{e-commerce} otvorenog koda napisana u Ruby on Rails radnom okruženju. 
Rad na Spreeu je počeo 2007. godine, a danas je jedna od najpopularnijih platformi za elektroničku trgovinu. 
Trenutno više od 45 tisuća trgovina koristi Spree \citep{spreeStorefront}.
\\
Odlike Spree rješenja su ogroman broj već implementiranih funkcija koje imaju sve moderne elektroničke trgovine, modularan izvorni tekst programa, lakoća dodavanja novih funkcionalnosti te izmjena i konfiguracija postojećih te brojne druge. 
Uz to, za gotovo svaki aspekt sustava dostupan je skup aplikacijsko programskih sučelja što pojednostavljuje integraciju sustava sa drugim tehnologijama (poput integracije sa mobilnim platformama).
\\
U 2013. godini objavljen je Spree Hub čija je funkcija rješavanje dodatnih poslova vezanih za elektroničku trgovinu. 
Neke odlike Spree Huba uključuju rješavanje logističkih problema (prijenos informacija o narudžbi iz trgovine u skladište), olakšana integracija sustava za računovodstvo, olakšana integraciju sustava za korisničku službu i brojne druge.
\\
Spree Hub se može integrirati skupa sa postojećim elektroničkim trgovinama i pri tome je bitno napomenuti da one ne moraju biti temeljene na Spree Commerce rješenju, već mogu biti temeljene na bilo kojoj drugoj platformi elektroničke trgovine (npr. Magento, Shopify i slične). 

\begin{figure}[H]
    \centering
    \includegraphics[scale=0.60]{pictures/spree_lavazza}
    \caption{Lavazza Store - jedna od mnogobrojnih elektroničkih trgovina koje koriste Spree}
    \label{pic:spreeLavazza}
\end{figure}

Spree je podijeljen na nekoliko ključnih komponenti (tj. Ruby gemova):
\begin{description}[style=nextline]
    \item[spree\_api] Skup aplikacijsko programskih sučelja koji zadovoljavaju REST \engl{Representational State Transfer} principe.
    \item[spree\_frontend] Komponente korisničkog sučelja.
    \item[spree\_backend] Komponente korisničkog sučelja administratora aplikacije.
    \item[spree\_cmd] Komponente sučelja naredbenog retka \engl{command-line interface, CLI}.
    \item[spree\_core] Ključne komponente, sadrži veliku većinu poslovne logike.
    \item[spree\_sample] Primjeri podataka.
\end{description}

Ako programer želi dodati neku od Spree komponenti u svoju postojeću aplikaciju, sve što treba napraviti je dodati željene komponente u Gemfile\footnote{datoteka koja sadrži popis svih Ruby gemova koji se koriste u projektu} te pokrenuti instalaciju.
\\
Najkraći način za instalaciju Spree gema, stvaranje novoga Ruby on Rails projekta i dodavanja Spreea u novostvoreni projekt prikazan je u nastavku (pretpostavlja se da su Ruby i Ruby on Rails prethodno instalirani i ispravno konfigurirani).

% caption=Instalacija Spree projekta, captionpos=t]
\begin{lstlisting}[language=bash]
$ gem install spree
$ rails _4.0.5_ new moj_projekt
$ spree install moj_projekt
\end{lstlisting}

Spree projekt se pokreće tako da se pozicioniramo u direktorij projekta i tamo pokrenemo server:
\begin{lstlisting}[language=bash]
$ cd moj_projekt
$ rails server
\end{lstlisting}

\begin{figure}[H]
    \centering
    \includegraphics[scale=0.66]{pictures/spree-small}
    \caption{Početni ekran Spree korisničkog sučelja}
    \label{pic:spree}
\end{figure}

\begin{figure}[H]
    \centering
    \includegraphics[scale=0.66]{pictures/spree-admin-small}
    \caption{Početni ekran Spree administratorskog sučelja}
    \label{pic:spreeAdmin}
\end{figure}

%dijelovi spreea
%Glavna komponenta bilo kojega sustava elektroničke trgovine, a tako i Spreea, je proizvod.
%\\
%U Spreeu svaki proizvod može imati varijante koje se međusobno razlikuju po jednom ili više svojstava (npr. veličini ili boji). 
%Također 

\subsection{Prilagođavanje i nadogradnja}

Modularna arhitektura Spree platforme omogućava jednostavno izmjenu, uklanjanje ili dodavanje novih funkcija u sustav, a kako bi korisničko sučelje bilo optimalno prikazano krajnjim korisnicima Spree nudi i osjetljivi dizajn. 
Postojeći dizajn je također modularan i željeno korisničko sučelje se može jednostavno ostvariti.
\\
Razvijatelji Spree sustava ne preporučuju prethodno spomenuti način prilagođavanja Spreea svojim potrebama, odnosno ne preporuča se izravna izmjena postojećeg izvornog tekst Spree programa \citep{spreeExt}. 
Preporučeni način prilagođavanja i nadogradnje Spree sustava je korištenjem takozvanih Spree ekstenzija.
\\
Spree ekstenzija nije ništa drugo nego zasebni Ruby gem kojega se dodaje u Gemfile postojećeg projekta. 
Primjer stvaranja i instaliranja Spree ekstenzije naveden je u nastavku.
\\

Stvaranje Spree ekstenzije:
\begin{lstlisting}[language=bash]
$ spree extension moja_ekstenzija
\end{lstlisting}

Za dodavanje ekstenzije u postojeći Spree projekt potrebno je dodati ime i putanju ekstenzije u Gemfile projekta:
\begin{lstlisting}[language=ruby]
gem 'moja_ekstenzija', path: '../moja_ekstenzija'
\end{lstlisting}

I nakon toga instalirati ekstenziju:
\begin{lstlisting}[language=bash]
$ bundle install
$ bundle exec rails g moja_ekstenzija:install
\end{lstlisting}

Nadogradnja modela i kontrolera je krajnje jednostavna i ne zahtjeva ponovno pisanje postojećih metoda pojedinih razreda, već se pišu samo nove metode ili metode čiji se sadržaj mijenja.
\\
Za nadogradnju postojećih pogleda Spree razvojni tim je razvio Deface. 
Deface je biblioteka za jednostavno nadograđivanje, izmjenu i uklanjanje komponenti iz postojećih predložaka. 
Instaliranjem Spree gema, automatski se instalira i Deface. 
Korištenjem Deface biblioteke programer ne mora kompletno mijenjati sve predloške koje njegova ekstenzija nadograđuje, već na elegantan način može dohvaćati pojedine komponente predloška i mijenjati njihov sadržaj.
\\
Prednost ekstenzija nad izravnom izmjenom izvornog teksta programa je ta što se ostvaruje bolja modularnost, olakšano je ponovno korištenje ekstenzija u drugim projektima te održavanje i nadogradnja sustava sa novim verzijama Spreea je puno jednostavnija. 
Uz to, Spree tim održava i službeni repozitorij\footnote{\url{http://spreecommerce.com/extensions}} gdje se može pronaći veliki broj besplatnih ekstenzija.
\\

Za integraciju iBeacon funkcionalnosti u Spree projekt razvijena je \textit{Spree Taxon Webservice} ekstenzija koja svakoj kategoriji proizvoda dodaje \textit{reference id} atribut. 
Na taj način će se ovisno o vrijednostima identifikatora obližnjeg odašiljača iBeacon prikazivati određena kategorija proizvoda.
\\

Sada mobilni uređaj nakon identifikacije obližnjeg odašiljača može poslati poslužitelju HTTP zahtjev sa pripadnim sadržajem:
\begin{lstlisting}
GET /api/products?q[taxons_ref_id_eq]=beaconId
\end{lstlisting}

U slučaju da je detektirano više odašiljača sadržaj zahtjva je:
\begin{lstlisting}
GET /api/products?q[taxons_ref_id_in][]=beaconId1
	&q[taxons_ref_id_in][]=beaconId2
\end{lstlisting}

Poslužitelj će kao odgovor poslati JSON objekt sa proizvodima koje pripadni odašiljač (ili odašiljači) referencira.

\begin{figure}[H]
    \centering
    \includegraphics[scale=0.37]{pictures/spree_taxon_webservice}
    \caption{Sučelje za izmjenu \textit{reference id} atributa kategorije proizvoda}
\end{figure}

\subsection{Instalacija i pokretanje poslužiteljske aplikacije}

Prije instaliranja i podešavanja poslužiteljske aplikacije potrebno je imati ispravno instaliran i konfiguriran Ruby te instalirane Ruby on Rails, Spree i Bundler Ruby gemove. 
Nakon toga se može stvoriti novi Ruby on Rails projekt i u njega ugraditi funkcionalnost Spree sustava:
\begin{lstlisting}[language=bash]
$ rails new spree_server_app
$ spree install spree_server_app
$ cd spree_server_app
\end{lstlisting}

Nakon toga je potrebno dodati ekstenziju Spree Taxon Webservice u Gemfile projekta:
\begin{lstlisting}[language=ruby]
gem 'spree_taxon_webservice', path: '../spree_taxon_webservice'
\end{lstlisting}

I instalirati dotičnu ekstenziju:
\begin{lstlisting}[language=bash]
$ bundle install
$ bundle exec rails g spree_taxon_webservice:install
\end{lstlisting}

Nakon toga, potrebno je dodati mehanizam CORS \engl{cross-origin resource sharing} u aplikaciju, kako bi se sadržaj poslužiteljske aplikacije mogao slati vanjskim aplikacijama (tj. aplikacijama iz druge domene). 
Najlakši način za rješavanje dotičnog problema je instalacija \textit{Rack CORS Middleware}\footnote{\url{https://github.com/cyu/rack-cors}} gema:
\begin{lstlisting}[language=bash]
$ gem install rack-cors
\end{lstlisting}

Dodavanje dotičnog gema u Gemfile projekta:
\begin{lstlisting}[language=ruby]
gem 'rack-cors', :require => 'rack/cors'
\end{lstlisting}

I konačno, omogućiti \textit{cross-origin} zahtjeve u projektu dodavanjem sljedećeg odsječka u \texttt{Application} razred koji se nalazi u \texttt{app/config/application.rb} datoteci:
\begin{lstlisting}[language=ruby]
config.middleware.use Rack::Cors do
  allow do
    origins '*'
    resource '*', headers: :any, methods: [:get, :post, :options]
  end
end
\end{lstlisting}

Ukoliko su svi prethodni koraci uspješno izvedeni, aplikacija se može pokrenuti sa sljedećom naredbom:
\begin{lstlisting}[language=bash]
$ rails server
\end{lstlisting}


\section{Klijentska aplikacija}

Za prevođenje neke od klijentskih aplikacija potrebno je imati instaliranu i konfiguriranu biblioteku za razvoj programske podrške željene mobilne platforme te Cordova sučelje naredbenog retka.
\\
Za instalaciju biblioteke za razvoj programske podrške dostupne su službene Cordova upute \citep{cordovaGuide}, a mogu se koristiti i službene upute proizvođača mobilne platforme.
\\
Cordova sučelje naredbenog retka može se instalirati sljedećom naredbom (pretpostavlja se da je \texttt{npm} instaliran i ispravno konfiguriran):
\begin{lstlisting}[language=bash]
$ sudo npm install -g cordova
\end{lstlisting}

Pojedina aplikacija se pokreće tako da se pozicioniramo u direktorij aplikacije (npr. \texttt{~/dev/cordova/experimental-app}) i  upišemo:
\begin{lstlisting}[language=bash]
$ cordova run android # za pokretanje na android uredaju
\end{lstlisting}

\subsection{Eksperimentalna aplikacija}

Za istraživanje i ispitivanje odašiljača iBeacon razvijena je posebna aplikacija. 
Aplikacija detektira obližnje odašiljače iBeacon, obrađuje primljene \textit{advertising} pakete i ispisuje osnovne informacije o odašiljaču, a može ispisati sve usluge (servise), karakteristike i opisnike odašiljača. 
Uz to, pomoću aplikacije mogu se prikupljati podaci od određenog odašiljača i ta funkcionalnost se može iskoristiti za kalibriranje određenog odašiljača, a i za izračun određenih informacija (poput srednje vrijednosti, standardne devijacije te minimalne i maksimalne jakosti primljenog signala). 
Također, određeni odašiljači mogu se blokirati tako da se informacije o njima ne prikazuju korisniku te u aplikaciju su ugrađene funkcije \eqref{eq:procjenaUdaljenosti} i \eqref{eq:rnFunkcija} koje se mogu iskoristiti za (nepouzdanu) procjenu udaljenosti.

\begin{figure}[H]
    \centering
    \begin{subfigure}[b]{0.45\textwidth}
        \centering
        \includegraphics[scale=0.15]{pictures/experimental1}
        \caption{Skeniranje obližnjih odašiljača}
        \label{fig:exp1}
    \end{subfigure}
     ~
     \begin{subfigure}[b]{0.45\textwidth}
        \centering
        \includegraphics[scale=0.15]{pictures/experimental2}
        \caption{Prikaz informacija o odašiljaču}
        \label{fig:exp2}
    \end{subfigure}
    
    \caption{Dio ostvarenih pogleda eksperimentalne aplikacije}
\end{figure}

\begin{figure}[H]
    \begin{subfigure}[b]{0.45\textwidth}
        \centering
        \includegraphics[scale=0.15]{pictures/experimental3}
        \caption{Prikupljanje podataka od odašiljača}
        \label{fig:exp3}
    \end{subfigure}
    ~
    \begin{subfigure}[b]{0.45\textwidth}
        \centering
        \includegraphics[scale=0.15]{pictures/experimental4}
        \caption{Procjena udaljenosti}
        \label{fig:exp4}
    \end{subfigure}
    
    \caption{Dio ostvarenih pogleda eksperimentalne aplikacije}
\end{figure}

\subsection{Mobilna aplikacija Spree}

U mobilnu aplikaciju Spree ugrađene su osnovne funkcionalnosti bilo koje elektroničke trgovine, poput pregledavanja proizvoda i kategorija proizvoda, košarica za kupnju te sam proces naplate. 
Svi elementi trgovine (popis proizvoda, kategorija i slično) dohvaćaju se sa poslužitelja na kojemu je pokrenuta Spree aplikacija obrađena u potpoglavlju \ref{sec:server}.
\\
Uz to u aplikaciju je integrirana i Cordova BLE biblioteka te BLE funkcionalnost.
Ugrađene su dvije BLE funkcije:
\begin{enumerate}
    \item Uređaj nekoliko sekundi skenira obližnje odašiljače iBeacon te od poslužitelja dohvaća sadržaj kojega referencira najbolji odašiljač.
    \item Uređaj nekoliko sekundi skenira obližnje odašiljače iBeacon te na temelju dva najbliža odašiljača od poslužitelja dohvaća referencirani sadržaj.
\end{enumerate}

Prva funkcionalnost može se iskoristiti u slučajevima gdje je velika trgovina raspoređena u manje odjele, a jedan odašiljač pokriva jedan odjel. 
Na primjer, trgovina može biti podijeljena na odjel sa bijelom tehnikom, odjel sa prehrambenim proizvodima i ostale odjele. 
\\
Druga funkcionalnost može se iskoristiti kod svakodnevnih trgovina gdje se proizvodi nalaze na policama, po jedna kategorija proizvoda sa svake strane police.

\begin{figure}[H]
    \centering
    \includegraphics[scale=0.85]{pictures/BLE-funkcionalnost}
    \caption{Primjer korištenja druge funkcionalnosti}
    \label{pic:bleShop}
\end{figure}
Na slici \ref{pic:bleShop} pravokutnici predstavljaju police gdje se na desnoj strani lijeve police nalaze torbe, lijevoj strani srednje police šalice, desnoj strane srednje police čaše, a na lijevoj strani desne police obuća. 
Na svakoj polici se nalazi jedan odašiljač iBeacon. 
Kupac sa mobilnim uređajem se nalazi između desne i srednje police, s tim da mu je bliža srednja polica, stoga će mu aplikacija prikazati kategoriju sa čašama.
\\

U slučaju da koristimo prvu funkcionalnost, nakon skeniranja odašiljača mobilna aplikacija će poslužitelju poslati sljedeći zahtjev:
\begin{lstlisting}
GET /api/products?q[taxons_ref_id_eq]=bestBeaconMinorValue
\end{lstlisting}
gdje parametar \textit{bestBeaconMinorValue} označava \textit{minor value} identifikator najboljeg (tj. najbližeg) odašiljača.

U slučaju da koristimo drugu funkcionalnost, nakon skeniranja odašiljača identifikatori dva najbolja odašiljača se na neki način moraju mapirati u broj koji će poslati poslužitelju:
\begin{lstlisting}
GET /api/products?q[taxons_ref_id_eq]=referenceId
\end{lstlisting}
Parametar \textit{referenceId} označava mapiranu vrijednost.

Kada se pokrene BLE funkcionalnost aplikacije, aplikacija skenira sve obližnje odašiljače te za svaki novopronađeni odašiljač stvara dinamičko polje u koju se pohrani jakost primljenog signala odašiljača (tj. RSSI odašiljača). 
Kada se opet primi signal od istoga odašiljača, nova jakost signala se dodaje u istu dinamičku listu. 
Pošto pojedini odašiljač šalje signal više puta u sekundi, nakon nekoliko sekundi svaka će lista imati veći broj elemenata.
Tada se pokreće sljedeći algoritam:
\begin{itemize}
    \item iz svih dinamičkih polja izbaci 10\% marginalnih vrijednosti
    \item sortiraj odašiljače tako da se na prvim mjestima nalaze odašiljači čija dinamička polja imaju najvišu srednju vrijednost
\end{itemize}
Izvorni tekst programa je prikazan u nastavku.

\begin{lstlisting}[language=java, morekeywords={var,function}]
// pomocna funkcija za izracun srednje vrijednosti
MathService.mean = function (data) {
  sum = 0;
  angular.forEach(data, function (val) {
    sum += val;
  });
  return sum / data.length;
};

// pomocna funkcija za uklanjanje marginalnih vrijednosti
MathService.removeMarginals = function (data, percentage) {
  percentage = angular.isDefined(percentage) ? percentage : 0.9;
  var average = this.mean(data);
  
  // sortiranje tako da se na pocetku nalaze vrijednosti
  // najblize srednjoj vrijednosti
  data.sort(function (a, b) {
  var distance = Math.abs(a - average) - Math.abs(b - average);
      if (distance < 0) {
        return -1;
      } else if (distance > 0) {
        return 1;
      }
      return 0;
    }
  });
  // broj elemenata koji se uklanja
  var numOfElToRm = Math.floor((1 - percentage) * data.length);
  // uklanjanje elemenata
  data.splice(data.length - numOfElToRm, numOfElToRm);
};

// uklanjanje marginalnih vrijednosti svih odasiljaca
angular.forEach(devices, function (device) {
  MathUtilities.removeMarginals(device.data, 0.9);
  device.averageRSSI = MathUtilities.mean(device.data);
  device.data = [];
});

// sortiranje odasiljaca
devices.sort(function (a, b) {
    if (a.averageRSSI < b.averageRSSI) {
      return 1;
    } else if (a.averageRSSI > b.averageRSSI) {
      return -1;
    }
    return 0;
  }
});
\end{lstlisting}

Ukoliko se koristi prva funkcionalnost, poslužitelju se šalje zahtjev za podacima i u parametrima se navodi \textit{minor value} identifikator prvog (najboljega) odašiljača. 
Ukoliko se koristi druga funkcionalnost, \textit{minor value} identifikatori dva najbolja odašiljača se mapiraju u broj koji se šalje poslužitelju prilikom dohvata podataka. 
Nakon toga se prazne liste svih odašiljača te se ciklus skeniranja obližnjih uređaja nastavlja.

\begin{figure}[H]
    \begin{subfigure}[b]{0.45\textwidth}
        \centering
        \includegraphics[scale=0.15]{pictures/spreeMobile1}
        \caption{Prikaz jedne kategorije proizvoda}
        \label{fig:spreeMobile1}
    \end{subfigure}
    ~
    \begin{subfigure}[b]{0.45\textwidth}
        \centering
        \includegraphics[scale=0.15]{pictures/spreeMobile2}
        \caption{Prikaz glavnog menija}
        \label{fig:spreeMobile2}
    \end{subfigure}
    
    \caption{Mobilna aplikacija Spree}
\end{figure}