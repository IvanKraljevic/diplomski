\begin{sazetak}

Tehnologije navigacije i pozicioniranja koje se oslanjaju na udaljene satelite (npr. GPS tehnologija) nisu pogodne za korištenje u zatvorenim prostorima. 
Stoga su za utvrđivanje lokacije u zatvorenom prostoru potrebne nove i drugačije metode. 
Kako ne postoji nikakav \textit{de facto} standard, danas postoji niz različitih rješenja. 
Uvođenjem Bluetooth 4.0 specifikacije i tehnologije Bluetooth Low Energy dolazi do razvoja niza jeftinih uređaja koji se mogu iskoristiti za rješavanje problema određivanja lokacije. 
Kako na signal u zatvorenom prostoru djeluje veliki broj smetnji određivanje lokacije pomoću BLE tehnologije nije precizno i preporuča se korištenje samo za određivanje okvirne lokacije. 
Unatoč tome, ono se može primijeniti u gotovo svim situacijama gdje se sadržaj mijenja ovisno o kontekstu prostora stoga se ono danas sve više integrira u mobilne aplikacije. 
%Razvojem radnih okruženja za izradu hibridnih mobilnih aplikacija, izrada mobilnih aplikacija za više platformi postaje jednostavnija, a

\kljucnerijeci{Bluetooth, BLE, mikrolokacija, iBeacon, Android, iOS, Apache Cordova, Ruby on Rails, Ruby, JavaScript, razvoj mobilnih aplikacija}
\end{sazetak}

\engtitle{Determining a micro-location of a mobile device}
\begin{abstract}
Positioning technologies that receive signals from distant satellites (such as GPS) are not suitable indoors. 
Because of that, the indoor navigation and positioning problem requires a new and different approach.
With the adoption of the Bluetooth 4.0 specification and the Bluetooth Low Energy technology, a large number of cheap Bluetooth Smart Ready devices that could be used for solving the indoor positioning problem has been manufactured. 
The radio signal that propagates in a indoor environment is affected by a large number of factors and because of that, using BLE technology for precise indoor positioning isn't appropriate.
Despite of that, it can be used in almost all situations where the content changes depending on the context of the environment so today BLE is being increasingly integrated into mobile applications.

\keywords{Bluetooth, BLE, mikrolokacija, iBeacon, Android, iOS, Apache Cordova, Ruby on Rails, Ruby, JavaScript, mobile development}
\end{abstract}