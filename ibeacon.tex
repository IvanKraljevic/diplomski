\chapter{Odašiljači iBeacon}

Odašiljači iBeacon su jeftini uređaji, niske potrošnje energije koji korištenjem BLE tehnologije obavještavaju obližnje uređaje o svojoj prisutnosti. 
Obližnji uređaji (poput mobilnih uređaja i tableta) se mogu pretplatiti na notifikacije odašiljača te mogu primati razne sadržaje (poput teksta, slika ili URL adresa) od njih. 
Krajem 2013. godine iBeacon tehnologiju patentirala je američka multinacionalna korporacija Apple Inc.
\\

iBeacon se može konfigurirati tako da se sadržaj šalje samo kad se uređaj približi odašiljaču na određenu udaljenost. 
Pri tome su definirana tri parametra udaljenosti: neposredna \textit{immediate}, mala \textit{near} i velika \textit{far} udaljenost. 
Neposredna udaljenost je do nekoliko centimetara, mala je do nekoliko metara, dok je velika udaljenost iznad deset metara. 
Ove vrijednosti nisu precizno specificirane jer ovisno točne vrijednosti ovise o stvarnim uvjetima u kojima su odašiljači postavljeni (zbog ometanja signala u zatvorenom prostoru).
\\

Neki od zanimljivih načina primjene iBeacon tehnologije su u muzejima, trgovinama, bolnicama i ostalim ustanovama gdje se sadržaj mijenja ovisno o položaju u prostoriji. 
Posjetitelj muzeja može na na mobilni uređaj ili tablet primiti sadržaj o objektu kojega trenutno promatra (npr. informacije o skulpturi ispred), uz to, mogu se pratiti koji su objekti ili egzibicije najgledaniji i slično. 
U bolnici se može primjeniti na način da kada se liječnik približi pacijentovoj sobi ili krevetu dobije sve podatke o njemu, od povijesti bolesti do trenutne dijagnoze. 
U trgovini se može iskoristiti tako da se kupca obavijesti o predmetima na popustu u blizi. 
Također, iBeacon odašiljači mogu se iskoristiti i kao sustav plaćanja na sličan način kako se i NFC\footnote{\textit{Near field communication}} tehnologija koristi.

\section*{Kontakt.io Beacon}

Glavne značajke uređaja su odašiljanje podatkovnih paketa korištenjem BLE tehnologije te kompatibilnost sa svim uređajima koji podražavaju Bluetooth 4.0. 
Uz to, uređaji se mogu jednostavno klonirati i nadograditi, imaju visoku razinu sigurnosti te malu energetsku potrošnju.
\\
Pojedini uređaj ima nekoliko konfigurabilnih parametara: ime uređaja, \textit{proximityUUID} \engl{Universally unique identifier}, \textit{major} i \textit{minor} vrijednosti, \textit{transmisssion power level} te interval odašiljanja poruka.
\\
Uređaj napaja jedna CR2477 baterija s kojom uređaj može konstantno raditi više od 24 mjeseca. %TODO referencirati službenu kontakt-io dokumentaciju
Raspon odašiljanja ovisi o \textit{transmisssion power level} te okolini u kojoj je postavljen (zbog difrakcije i apsorpcije signala).

\subsection*{Struktura paketa}
Uređaj odašilje dvije vrste paketa podataka: \textit{advertising} i \textit{scan response}.
\\
Tokom rada uređaj kontinuriano odašilje \textit{advertising} pakete i na taj način se "oglašava" okolnim uređajima. Drugi tip paketa, \textit{scan response}, šalje se odmah nakon \textit{advertising} paketa i sadrži dodatne informacije o odašiljaču, poput imena odašiljača, stanja baterije i slično.
\\

Struktura \textit{advertising} paketa je prikazana u tablici \ref{tbl:advertising}.
\\
\begin{table}
\label{tbl:advertising}
\begin{tabular}{|c|c|c|c|}
\hline 
Byte & Pretpostavljena (engl. default) vrijednost & Opis & Svojstvo \\ 
\hline 
1 & 02 & Duljina podataka & constant preamble \\ 
\hline 
2 & 01 & Tip podataka - flags & constant preamble \\ 
\hline 
3 & 06 & LE, BF/EDR zastavice & constant preamble \\ 
\hline 
4 & 1a & Duljina podataka* & constant preamble \\ 
\hline 
5 & ff & Tip podataka* & constant preamble \\ 
\hline 
6 & 4c & Podaci o proizvođaču & constant preamble \\ 
\hline 
7 & 00 & Podaci o proizvođaču & constant preamble \\ 
\hline 
8 & 02 & Podaci o proizvođaču & constant preamble \\ 
\hline 
9 & 15 & Podaci o proizvođaču & constant preamble \\ 
\hline 
10 & f7 & Proximity UUID 1. bajt & user UUID \\ 
\hline 
11 & 82 & Proximity UUID 2. bajt & user UUID \\ 
\hline 
12 & 6d & Proximity UUID 3. bajt & user UUID \\ 
\hline 
13 & a6 & Proximity UUID 4. bajt & user UUID \\ 
\hline 
14 & 4f & Proximity UUID 5. bajt & user UUID \\ 
\hline 
15 & a2 & Proximity UUID 6. bajt & user UUID \\ 
\hline 
16 & 4e & Proximity UUID 7. bajt & user UUID \\ 
\hline 
17 & 98 & Proximity UUID 8. bajt & user UUID \\ 
\hline 
18 & 80 & Proximity UUID 9. bajt & user UUID \\ 
\hline 
19 & 24 & Proximity UUID 10. bajt & user UUID \\ 
\hline 
20 & bc & Proximity UUID 11. bajt & user UUID \\ 
\hline 
21 & 5b & Proximity UUID 12. bajt & user UUID \\ 
\hline 
22 & 71 & Proximity UUID 13. bajt & user UUID \\ 
\hline 
23 & e0 & Proximity UUID 14. bajt & user UUID \\ 
\hline 
24 & 89 & Proximity UUID 15. bajt & user UUID \\ 
\hline 
25 & 3e & Proximity UUID 16. bajt & user UUID \\ 
\hline 
26 & xx & Major 1. bajt & major value \\ 
\hline 
27 & xx & Major 2. bajt & major value \\ 
\hline 
28 & xx & Minor 1. bajt & minor value \\ 
\hline 
29 & xx & Minor 2. bajt & minor value \\ 
\hline 
30 & b3 & Jakost signala & signal power value \\ 
\hline 
\end{tabular} 
\end{table}
\textit{Advertising} paketi zadovoljavaju Apple iBeacon\textsuperscript{\texttrademark} standard.
\\
Prvih devet bajtova paketa su unaprijed poznate konstante (poput informacije o proizvođaču). Njih slijedi šesnaest bajtova koji čine proximity UUID, dva bajta za \textit{major} vrijednost, dva bajta za \textit{minor} vrijednost, a posljednji bajt predstavlja RSSI vrijednost (engl. \textit{Recived Signal Strength Indication}) izmjeren jedan metar od Beacon uređaja.
\\

Struktura \textit{scan response} paketa je prikazana u tablici \ref{tbl:scanResponse}.

\begin{table}
    \label{tbl:scanResponse}
\end{table}
% TODO ispisati txPowerValues