\chapter{Zaključak}
\label{chap:zakljucak}

Unatoč tome što je razvijen cijeli niz novih metoda za rješavanje problema navigacije i pozicioniranja u zatvorenom prostoru, još nema općeprihvaćenih rješenja. 
\\
Korištenje BLE tehnologije i iBeacon odašiljača za rješavanje problema procjene udaljenosti u zatvorenom prostoru nije precizno ni pouzdano iz razloga što na signal koji putuje zatvorenim prostorom djeluje veliki broj smetnji koje se često ne mogu smanjiti ni otkloniti. 
%Na signal kojega odašilje iBeacon, odnosno BLE, odašiljač u zatvorenom prostoru djeluje veliki broj smetnji stoga pouzdana i precizna procjena udaljenosti na temelju jačine primljenog signala nije moguća. 
Bez obzira na to, iBeacon odašiljači mogu se primijeniti u gotovo svim situacijama gdje se sadržaj mijenja ovisno o kontekstu prostora. 
Iz tog razloga, iBeacon tehnologija se sve više integrira u mobilne aplikacije.
\\

U praktičnom dijelu rada prikazano je kako se iBeacon tehnologija može integrirati sa hibridnom mobilnom aplikacijom tako da prikazuje različite podatke ovisno o lokaciji mobilnog uređaja. 
%Prikazano je kako se pomoću HTML, CSS i JavaScript programskog jezika mogu razviti hibridne mobilne aplikacije te 