\chapter{Radno okruženje Apache Cordova}
\label{chap:radnoOkruzenje}

Radno okruženje Apache Cordova je skup aplikacijsko programskih sučelja \engl{application programming interface, API} koji omogućavaju da razvijatelj mobilnih aplikacija pristupa osnovnim funkcijama mobilnoga uređaja, poput kamere, sustava za pohranu podataka i telefonskog imenika preko JavaScript jezika. 
U kombinaciji sa radnim okruženjima poput Sencha Touch, Dojo Mobile i Ionic aplikacije za pametne telefone mogu se razvijati korištenjem samo HTML, CSS i JavaScript programskog jezika.
\\

Korištenjem Apache Cordove programer je oslobođen pisanja aplikacija u nativnim jezicima uređaja (npr. Java za Android, Objective-C za iOS), već se koriste isključivo prethodno spomenute web tehnologije. 
Bez obzira na to što aplikacije nisu napisane u nativnim jezicima, Apache Cordova aplikacije se prevode i pakiraju pomoću SDK \engl{software development kit, SDK} željene platforme stoga se aplikacije mogu i postaviti na trgovine aplikacija \engl{app store} dotične platforme. 
\\

Cordova nudi skup uniformnih JavaScript biblioteka čije funkcije programer može pozivati. 
One imaju podršku za povezivanje sa specifičnim platformama. Cordova je trenutno dostupna za sljedeće platforme: Android, iOS, Blackberry,Windows Phone, Palm WebOS, Bada i Symbian.
\\

Za pristup \textit{Bluetooth Low Energy} funkcijama mobilnog uređaja korištena je \textbf{Cordova BLE} biblioteka\footnote{\url{https://github.com/evothings/cordova-ble}}, dok je oko dvije stotine dodatnih i besplatnih biblioteka dostupno na službenom repozitoriju\footnote{\url{http://plugins.cordova.io}}.
\\
Službene Cordova JavaScript biblioteke koje održava Cordova tim su:

\begin{description}[style=nextline]
	    \item[Battery Status]
	        Omogućava nadgledanje stanja baterije uređaja.
	    \item[Camera]
	        Omogućava pristup kameri uređaja.
	    \item[Contacts]
	        Omogućava pristup telefonskom imeniku uređaja.
	    \item[Device]
	        Omogućava pristup specifičnim informacijama uređaja (npr. ime uređaja, operacijski sustav).
	    \item[Device Motion (Accelerometer)]
	        Omogućava pristup senzoru ubrzanja (akcelerometar).
	    \item[Device Orientation (Compass)]
	        Omogućava pristup kompasu uređaja.
	    \item[Dialogs]
	        Omogućava korištenje sustava obavijesti uređaja.
	    \item[FileSystem]
	        Omogućava korištenje datotečnog sustava uređaja.
	    \item[FileTransfer]
	        Omogućava pristup sustavu za prijenos datoteka.
	    \item[Geolocation]
	        Omogućava pristup prema geolokacijskom sustavu. 
	    \item[Globalizationg]
	        Omogućava različite reprezentacije objekata ovisno o postavkama lokacije uređaja.
	    \item[InAppBrowser]
	        Omogućava otvaranje URL-ova u novoj instanci web-preglednika uređaja.
	    \item[Media]
	        Omogućava snimanje i reprodukciju audio datoteka.
	    \item[Media Capture]
	        Omogućava snimanje audio i video datoteka.
	    \item[Network Information (Connection)]
	        Omogućava pristup informacijama o stanju mreže uređaja.
	    \item[Splashscreen]
	        Omogućava manipuliranje početnog zaslona aplikacije.
	    \item[Vibration]
	        Omogućava korištenje mehanizma za vibriranje uređaja.
\end{description}


\section*{Radno okruženje Ionic}

Ionic je radno okruženje napisano sa HTML, CSS i JavaScript programskim jezikom čiji je cilj olakšati razvoj hibridnih mobilnih aplikacija\footnote{Aplikacije napravljene korištenjem web tehnologija. 
Pokreću se unutar posebnog spremnika uređaja te koriste funkcije web preglednika za prikaz HTML sadržaja i izvođenje JavaScript k\^oda} stoga je ono jako dobar izbor prilikom izrade Cordova aplikacija. 
Ionic je primarno okrenut prema olakšanju izrade korisničkog sučelja, odnosno nudi cijeli niz funkcija koje razvijatelju olakšavaju izradu velikih i složenih mobilnih aplikacija. 
U pozadini, Ionic koristi danas sve popularnije JavaScript \textit{frontend} radno okruženje AngularJS koje je namjenjeno izradi \textit{single-page} web aplikacija. 
Korištenjem AngularJSa u Ionic je dodan cijeli niz direktiva\footnote{Posebne oznake DOM elemenata koje obavještavaju AngularJS HTML prevoditelja da transformira elemente ili im doda novo ponašanje.}, filtera\footnote{Funkcije za obradu ili transformaciju podataka koji se prikazuju korisniku. 
Npr. mogu se koristiti za abecedni prikaz elemenata liste, prikaz teksta isključivo velikim slovom itd.}, servisa\footnote{Jedinstveni objekti \engl{singleton} koji se koriste za dijeljenje funkcija i resursa unutar aplikacije.} i drugih funkcija koje programeru rješavaju cijeli niz problema poput osjetljivog dizajna \engl{responsive design}, hvatanja raznih korisnikovih interakcija (dodiri s jedim ili više prstiju, povlačenje \engl{swipe} stavki, stezanju i širenju stavki \engl{pinch} i cijeli niz drugih). 
\\

Ionic se sastoji od dva temeljna dijela. 
Prvi dio čine CSS i Sass datoteke čija je svrha da razvijateljima olakšaju izradu vizualnoga dizajna aplikacije. 
Drugi dio čine JavaScript datoteke i HTML predlošci koji pojednostavljuju izradu aplikacija složene arhitekture te programerima nude cijeli niz pomoćnih funkcija.
\\

Ionic je relativno novo radno okruženje i u vrijeme pisanja ovog rada je u beta verziji. 
Unatoč tome, veliki broj funkcija već je implementiran i samo njihovim korištenjem mogu se napraviti velike, složene i vizualno privlačne mobilne aplikacije. 
Uz to, Ionic je izdan pod MIT licencom i njegov izvorni kod je javno dostupan na službenom Github repozitoriju\footnote{\url{https://github.com/driftyco/ionic}}.