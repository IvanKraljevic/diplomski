\chapter{Radno okruženje Apache Cordova}
\label{chap:radnoOkruzenje}

Radno okruženje Apache Cordova je skup aplikacijsko programskih sučelja \engl{application programming interface, API} koji omogućavaju da razvijatelj mobilnih aplikacija pristupa osnovnim funkcijama mobilnoga uređaja, poput kamere, sustava za pohranu podataka i telefonskog imenika preko JavaScript jezika. 
U kombinaciji sa radnim okruženjima poput Sencha Touch, Dojo Mobile i Ionic aplikacije za pametne telefone mogu se razvijati korištenjem samo HTML, CSS i JavaScript programskog jezika.
\\

Korištenjem Apache Cordove programer je oslobođen pisanja aplikacija u nativnim jezicima uređaja (npr. Java za Android, Objective-C za iOS), već se koriste isključivo prethodno spomenute web tehnologije. 
Bez obzira na to što aplikacije nisu napisane u nativnim jezicima, Apache Cordova aplikacije se prevode i pakiraju pomoću biblioteke za razvoj programske potpore \engl{software development kit, SDK} željene platforme stoga se aplikacije mogu postaviti na trgovine aplikacija \engl{app store} dotične platforme. 
\\

Cordova nudi skup uniformnih JavaScript biblioteka čije funkcije programer može pozivati. 
One imaju podršku za povezivanje sa specifičnim platformama. Cordova je trenutno dostupna za sljedeće platforme: Android, iOS, Blackberry,Windows Phone, Palm WebOS, Bada i Symbian.
\\

Za pristup \textit{Bluetooth Low Energy} funkcijama mobilnog uređaja korištena je \textbf{Cordova BLE} biblioteka\footnote{\url{https://github.com/evothings/cordova-ble}}, dok je oko dvije stotine dodatnih i besplatnih biblioteka dostupno na službenom repozitoriju\footnote{\url{http://plugins.cordova.io}}.
\\
Službene Cordova JavaScript biblioteke koje održava Cordova tim su:

\begin{description}[style=nextline]
	    \item[Battery Status]
	        Omogućava nadgledanje stanja baterije uređaja.
	    \item[Camera]
	        Omogućava pristup kameri uređaja.
	    \item[Contacts]
	        Omogućava pristup telefonskom imeniku uređaja.
	    \item[Device]
	        Omogućava pristup specifičnim informacijama uređaja (npr. ime uređaja, operacijski sustav).
	    \item[Device Motion (Accelerometer)]
	        Omogućava pristup senzoru ubrzanja (akcelerometar).
	    \item[Device Orientation (Compass)]
	        Omogućava pristup kompasu uređaja.
	    \item[Dialogs]
	        Omogućava korištenje sustava obavijesti uređaja.
	    \item[FileSystem]
	        Omogućava korištenje datotečnog sustava uređaja.
	    \item[FileTransfer]
	        Omogućava pristup sustavu za prijenos datoteka.
	    \item[Geolocation]
	        Omogućava pristup prema geolokacijskom sustavu. 
	    \item[Globalizationg]
	        Omogućava različite reprezentacije objekata ovisno o postavkama lokacije uređaja.
	    \item[InAppBrowser]
	        Omogućava otvaranje URL-ova u novoj instanci web-preglednika uređaja.
	    \item[Media]
	        Omogućava snimanje i reprodukciju audio datoteka.
	    \item[Media Capture]
	        Omogućava snimanje audio i video datoteka.
	    \item[Network Information (Connection)]
	        Omogućava pristup informacijama o stanju mreže uređaja.
	    \item[Splashscreen]
	        Omogućava manipuliranje početnog zaslona aplikacije.
	    \item[Vibration]
	        Omogućava korištenje mehanizma za vibriranje uređaja.
\end{description}

\section{Biblioteka Cordova BLE}

Neslužbena Cordova BLE biblioteka omogućava hibridnim mobilnim aplikacijama koje se temelje na Cordovi korištenje BLE funkcionalnosti mobilne platforme bez potrebe za pisanjem nativnog programskog kôda. 
Preko biblioteke moguće je korištenjem JavaScript programskog jezika skenirati obližnje BLE odašiljače, spajati se na njih, čitati servise, karakteristike i opisnike te pisati nove vrijednosti karakteristika i opisnika (ukoliko odašiljač to dopušta). 
Također, biblioteka omogućava da se aplikacija pretplati na obavijesti odašiljača te može resetirati Bluetooth sustav mobilnog uređaja. 
Resetiranje Bluetooth sustava osobito je korisno aplikacijama koje se pokreću na Android operacijskom sustavu iz razloga što raniji Android sustavi imaju popriličan broj pogrešaka u svojoj Bluetooth implementaciji. 
Dobar dio grešaka ispravlja Android 4.4.3 koji je objavljen početkom lipnja 2014. godine, no dotični operacijski sustav imaju imaju samo noviji moderniji uređaji stoga je korištenje funkcije resetiranja nužno kako bi se osigurao ispravan rad mobilne aplikacije i u nešto starijim Android uređajima.
\\

Biblioteka funkcionira na način da se razvijatelju aplikacije nudi skup uniformnih funkcija za pristup BLE funkcionalnosti mobilnog uređaja preko JavaScript programskog jezika. 
U pozadini dotične JavaScript funkcije pozivaju funkcije koje su napisane u nativnom jeziku platforme (tj. operacijskog sustava) mobilnog uređaja na kojem je aplikacija pokrenuta. 
Nadalje, dotične nativne funkcije pristupaju samoj BLE funkcionalnosti uređaja koje u konačnici obave zadani posao. 
Na kraju, nativne funkcije šalju odgovore JavaScript sloju preko prethodno predanih asinkronih povratnih\engl{callback} funkcija.
Konkretno, kompletan tok poziva određene funkcije prikazan je na sljedećem primjeru gdje se pokreće traženje obližnjih BLE odašiljača.
\\

Iz našeg JavaScript programa pozivamo funkciju za skeniranje obližnjih uređaja:
\begin{lstlisting}[language=java, morekeywords={var,function}]
// povratna funkcija koja se poziva kada se pronade odasiljac
var win = function(device) {
  //ucini nesto sa pronadenim uredajem
};

// povratna funkcija koja se poziva ukoliko dode do pogreske
var fail = function(error) {
  // obrada pogreske
};

// pokretanje skeniranja
evothings.ble.startScan(win, fail);
\end{lstlisting}

Pozvana funkcija poziva nativnu funkciju i prosljeđuje joj predane povratne funkcije:
\begin{lstlisting}[language=java, morekeywords={var,function}]
// ovo je zapravo prethodno pozvana evothings.ble.startScan funkcija
exports.startScan = function(win, fail) {
	exec(win, fail, 'BLE', 'startScan', []);
};
\end{lstlisting}

Na Android platformi gore navedena \texttt{exec} funckija je zapravo \texttt{execute} metoda razreda \texttt{BLE} koji nasljeđuje \texttt{CordovaPlugin} razred te implementira funkcionalnost \texttt{LeScanCallback} sučelja. 
\begin{lstlisting}[language=java]
public class BLE extends CordovaPlugin implements LeScanCallback {
  @Override
  public boolean execute(String action, CordovaArgs args, 
    final CallbackContext callbackContext) throws JSONException {
    if ("startScan".equals(action)) {
      // poziv startScan funckije napisane u nativnom programskom
      // jeziku Android platforme
      startScan(args, callbackContext); 
      return true;
    } else if (...) {
      ...
    }
}
...
}
\end{lstlisting}

Svaka Cordova biblioteka mora naslijediti razred \texttt{CordovaPlugin} i nadjačati\engl{override} neku od   izvršnih metoda, odnosno u ovom slučaju nasljeđuje se \texttt{execute} metoda. 
Sučelje \texttt{LeScanCallback} je nužno zbog pristupa BLE funkcionalnosti Android platforme.

Ako funkcija uspješno obradi zadatak pozvat će se povratna funkcija \texttt{win}, a ako je došlo do pogreške pozvati će se povratna funckija \texttt{fail}. 
Obe funkcije su pohranjene instancu \texttt{CallbackContext} razreda.
\\

U vrijeme pisanja rada Cordova BLE biblioteka nudi podršku samo prema iOS i Android operacijskim sustavima.

\section{Radno okruženje Ionic}

Ionic je radno okruženje napisano sa HTML, CSS i JavaScript programskim jezikom čiji je cilj olakšati razvoj hibridnih mobilnih aplikacija\footnote{Aplikacije napravljene korištenjem web tehnologija. 
Pokreću se unutar posebnog spremnika uređaja te koriste funkcije web preglednika za prikaz HTML sadržaja i izvođenje JavaScript k\^oda} stoga je ono jako dobar izbor prilikom izrade Cordova aplikacija. 
Ionic je primarno okrenut prema olakšanju izrade korisničkog sučelja, odnosno nudi cijeli niz funkcija koje razvijatelju olakšavaju izradu velikih i složenih mobilnih aplikacija. 
U pozadini, Ionic koristi danas sve popularnije JavaScript \textit{frontend} radno okruženje AngularJS koje je namjenjeno izradi \textit{single-page} web aplikacija. 
Korištenjem AngularJSa u Ionic je dodan cijeli niz direktiva\footnote{Posebne oznake DOM elemenata koje obavještavaju AngularJS HTML prevoditelja da transformira elemente ili im doda novo ponašanje.}, filtera\footnote{Funkcije za obradu ili transformaciju podataka koji se prikazuju korisniku. 
Npr. mogu se koristiti za abecedni prikaz elemenata liste, prikaz teksta isključivo velikim slovom itd.}, servisa\footnote{Jedinstveni objekti \engl{singleton} koji se koriste za dijeljenje funkcija i resursa unutar aplikacije.} i drugih funkcija koje programeru rješavaju cijeli niz problema poput osjetljivog dizajna \engl{responsive design}, hvatanja raznih korisnikovih interakcija (dodiri s jedim ili više prstiju, povlačenje \engl{swipe} stavki, stezanju i širenju stavki \engl{pinch} i cijeli niz drugih). 
\\

Ionic se sastoji od dva temeljna dijela. 
Prvi dio čine CSS i Sass datoteke čija je svrha da razvijateljima olakšaju izradu vizualnoga dizajna aplikacije. 
Drugi dio čine JavaScript datoteke i HTML predlošci koji pojednostavljuju izradu aplikacija složene arhitekture te programerima nude cijeli niz pomoćnih funkcija.
\\

Ionic je relativno novo radno okruženje i u vrijeme pisanja ovog rada je u beta verziji. 
Unatoč tome, veliki broj funkcija već je implementiran i samo njihovim korištenjem mogu se napraviti velike, složene i vizualno privlačne mobilne aplikacije. 
Uz to, Ionic je izdan pod MIT licencom i njegov izvorni kod je javno dostupan na službenom Github repozitoriju\footnote{\url{https://github.com/driftyco/ionic}}.