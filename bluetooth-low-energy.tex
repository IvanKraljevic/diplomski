\chapter{Tehnologija Bluetooth Low Energy}
\label{chap:ble}
    
Tehnologija Bluetooth je standard bežične komunikacije koji se koristi za razmjenu podataka na maloj udaljenosti. 
Bluetooth je razvijen 1994. godine u Ericssonu, a 1998. godine Ericsson, IBM, Intel, Nokia i Toshiba osnivaju posebno nadležno tijelo, Bluetooth Special Interest Group (SIG). 
Uloga nadležnog tijela je unaprijeđenje standarda, ispravna implementacija i licenciranje Bluetooth tehnologije.
\\

Glavne odlike Bluetooth tehnologije su niska cijena Bluetooth uređaja, niska potrošnja energije, niski domet, robusnost te korištenje na globalnoj razini. 
Bluetooth omogućava brzinu prijenosa reda veličine 1 Mbit/s te koristi nelicencirani frekvencijski pojas od 2.4 do 2.485 GHz, odnosno koristi ISM područje \engl{industrial, scientific and medical} koje je frekvencijski usklađeno na globalnoj razini. 
Uz to, Bluetooth nudi radijsku vezu prema drugim sustavima, uređaji različitih proizvođača su međusobno kompatibilni i dopuštena je komutacija paketa i kanala.
\\% \SI{1}{Mbit/s}  \SI{2.485}{GHz}

Sredinom 2010. godine Bluetooth SIG objavljuje Bluetooth 4.0 specifikaciju koja uključuje \textit{Classic Bluetooth}, \textit{Bluetooth high speed} i \textit{Bluetooth low energy} protokole. 
\textit{Bluetooth low energy} (u daljenjem tekstu BLE), poznat i pod nazivom \textit{Bluetooth Smart}, je tehnologija koja je optimizirana tako da ima veoma nisku potrošnju energije. 
Uz to, glavne odlike tehnologije su mogućnost višegodišnjeg rada na malom izvoru energije (poput \textit{button-cell} ili AAA baterije), mala veličina i niska cijena te kompatibilnost sa mobilnim uređajima, tabletima i računalima. 
\\
Za ugradnju BLE tehnologije u uređaje Bluetooth 4.0 specifikacija uvodi dva načina rada: \textit{single-mode} i \textit{dual-mode}. 
\textit{Single-mode} način rada obuhvaća integraciju samo BLE funkcionalnosti u kontroler, dok \textit{dual-mode} načina rada omogućava integraciju BLE funkcionalnost u standardni Bluetooth kontroler. 
Proizvođači uređaja imaju na raspolaganju te dvije opcije i pri tome je važno napomenuti da uređaji sa \textit{single-mode} načinom rada ne mogu komunicirati sa uređajima koji koriste klasični Bluetooth protokol.
\\

Danas se velika većina mobilnih uređaja proizvodi sa podrškom i za standardni Bluetooth i za BLE, tj. u uređaje se ugrađuje Bluetooth mikrokontroler sa \textit{dual-mode} načinom rada. 
Mobilni operacijski sustavi koji trenutno podržavaju BLE su:

\begin{itemize}
    \item Android 4.3 i noviji
    \item iOS 5 i noviji
    \item Windows Phone 8.1
    \item Blackberry 10
\end{itemize}
%Bluetooth SIG predviđa da će do 2018. godine devedeset posto Bluetooth uređaja na pametnim telefonima podržavati i BLE. 
%BLE je osmišljen tako da ga mogu koristiti uređaji sa malim napajanjem (npr. AAA ili CR2032 baterije). 
Bitno je napomenuti da BLE nije i ne pokušava biti optimizirana verzija \textit{Bluetooth classic} tehnologije, već cilja na sasvim nove načine primjene. 
Predviđene primjene su u sportu, zdravstvu, trgovini, turizmu, mjerenju udaljenosti i druge.


\section{Tehničke značajke}

\textit{Bluetooth Low Energy} tehnologija temelji se na profilima i \textit{Generic Attribute Profile} (u daljnjem tekstu GATT) specifikaciji. 
\\
BLE profili su posebne specifikacije koje definiraju servise koji se koriste u određenim scenarijima. 
Od proizvođača konkretnih uređaja se očekuje da zadovolje određen profil kako bi se osigurala međusobna kompatibilnost između raznih uređaja (npr. ako uređaj koji prati krvni tlak osobe zadovoljava HRP profil svi uređaji mogu uniformno čitati njegove servise). 
Popularniji standardizirani profili su:

\begin{itemize}
    \item HTP (\textit{Health Thermometer Profile}) - omogućava uređaju spajanje i interakciju sa termometrom
    \item GLP (\textit{Glucose Profile}) - omogućava uređaju spajanje i interakciju sa senzorom koji mjeri razinu glukoze u krvi osobe
    \item BLP (\textit{Blood Pressure Profile}) - omogućava uređaju spajanje i interakciju sa senzorom koji mjeri krvni tlak osobe
    \item HRP (\textit{Heart Rate Profile}) - omogućava uređaju spajanje i interakciju sa senzorom rada srca 
    \item FMP (\textit{Find Me Profile}) - definira ponašanje gdje pritisak gumba na jednom uređaju šalje obavijest drugom uređaju
    \item PXP (\textit{Proximity Profile}) - omogućava praćenje udaljenosti između dva uređaja
    \item LNP (\textit{Location and Navigation Profile}) - omogućava uređaju spajanje i interakciju sa senzorom navigacije
\end{itemize}
Ostali profili mogu se naći na službenim Bluetooth stranica\footnote{\url{https://developer.bluetooth.org/gatt/profiles/}}. Svi BLE profili su definirani na temelju GATT specifikacije.
\\

GATT specifikacija definira način na koji se šalju i primaju podaci kratke duljine (često zvani atributi) preko BLE veze. 
GATT koristi \textit{Attribute} protokol (u daljnjem tekstu ATT) koji ima sličnu funkcionalnost kao SDP\footnote{\textit{Service discovery protocol}, omogućava uređajima da međusobno saznaju koje servise podržavaju i koje parametre trebaju koristiti pri stvaranju podatkovne veze} protokol kod standardnog Bluetootha, samo što je optimiziran i pojednostavljen za korištenje u uređajima male potrošnje. 
GATT definira na koji način su ATT atributi grupirani tako da čine konkretne servise. 
Bitni koncepti kod GATT specifikacije su:

\begin{description}[style=nextline]
    \item[Klijent] 
        Uređaj koji pokreće komunikaciju te šalje zahtjeve i prima odgovore (npr. mobilni uređaj ili računalo).
    \item[Server] 
        Uređaj koji prima zahtjeve i šalje odgovore (npr. senzor temperature ili iBeacon odašiljač).
    \item[Servis] 
        Kolekcija povezanih karakteristika koje zajedno čine nekakvu funkciju (npr. \textit{Health Thermometer} servis uključuje                     karakteristike za vrijednost temperature, interval čitanja i mjernu jedinicu temperature). 
        Svaki servis može imati proizvoljan broj karakteristika.
    \item[Karakteristika\footnote{engl. \textit{Characteristic}}] 
        Vrijednost koja se izmjenjuje između klijenta i servera (npr. krvni tlak osobe ili trenutno stanje baterije). 
        Svaka karakteristika može imati proizvoljan broj opisnika.
    \item[Opisnik\footnote{engl. \textit{Descriptor}}] 
        Vrijednost koja pobliže opisuje neku karakteristiku (npr. minimalna i maksimalna vrijednost karakteristike ili mjerna jedinica karakteristike).
\end{description}