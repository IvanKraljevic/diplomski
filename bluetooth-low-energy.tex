\chapter{Tehnologija Bluetooth Low Energy}

Tehnologija Bluetooth je standard bežične komunikacije koji se koristi za razmjenu podataka na maloj udaljenosti. 
Bluetooth je razvijen 1994. godine u Ericssonu, a 1998. godine Ericsson, IBM, Intel, Nokia i Toshiba osnivaju posebno nadležno tijelo, Bluetooth Special Interest Group (SIG). 
Uloga nadležnog tijela je unaprijeđenje standarda, ispravna implementacija i licenciranje Bluetooth tehnologije.
\\

Glavne odlike Bluetooth tehnologije su niska cijena Bluetooth uređaja, niska potrošnja energije, niski domet, robusnost te korištenje na globalnoj razini. 
Bluetooth omogućava brzinu prijenosa reda veličine \SI{1}{Mbit/s} te koristi nelicencirani frekvencijski pojas od $2.4$ do \SI{2.485}{GHz}, odnosno koristi ISM područje \engl{industrial, scientific and medical} koje je frekvencijski usklađeno na globalnoj razini. 
Uz to, Bluetooth nudi radijsku vezu prema drugim sustavima, uređaji različitih proizvođača su međusobno kompatibilni i dopuštena je komutacija paketa i kanala.
\\

Sredinom 2010. godine Bluetooth SIG objavljuje Bluetooth 4.0 specifikaciju koja uključuje \textit{Classic Bluetooth}, \textit{Bluetooth high speed} i \textit{Bluetooth low energy} protokole.
\\
\textit{Bluetooth low energy} (u daljenjem tekstu BLE), poznat i pod nazivom \textit{Bluetooth Smart}, je tehnologija koja je optimizirana tako da ima veoma nisku potrošnju energije. 
Glavne odlike tehnologije su izuzetno niska potrošnja energije, mogućnost višegodišnjeg rada na malom izvoru energije (poput \textit{button-cell} ili AAA baterije), mala veličina i niska cijena, kompatibilnost sa mobilnim uređajima, tabletima i računalima. 
\\
Za ugradnju BLE tehnologije u uređaje Bluetooth 4.0 specifikacija uvodi dva načina rada: \textit{single-mode} i \textit{dual-mode}. 
\textit{Single-mode} način rada obuhvaća integraciju samo BLE funkcionalnosti na kontroler, dok \textit{dual-mode} načina rada omogućava integraciju BLE funkcionalnost u standardni \textit{Classic Bluetooth} kontroler. 
Proizvođači uređaja imaju na raspolaganju te dvije opcije i pri tome je važno napomenuti da uređaji sa \textit{single-mode} načinom rada ne mogu komunicirati sa uređajima koji koriste klasični Bluetooth protokol.
\\

Bluetooth SIG predviđa da će do 2018. godine devedeset posto Bluetooth uređaja na pametnim telefonima podržavati i BLE. 
%BLE je osmišljen tako da ga mogu koristiti uređaji sa malim napajanjem (npr. AAA ili CR2032 baterije). 
Bitno je napomenuti da BLE nije i ne pokušava biti optimizirana verzija \textit{Bluetooth classic} tehnologije, već cilja na sasvim nove načine primjene. 
Predviđene primjene su u sportu, zdravstvu, trgovini, turizmu, mjerenju udaljenosti i druge.


\section*{Tehničke značajke}

\textit{Bluetooth Low Energy} tehnologija temelji se na profilima i \textit{Generic Attribute Profile} (GATT) specifikaciji. 
GATT definira način na koji se šalju i primaju podaci kratke duljine (često zvani atributi) preko BLE veze.
\\
GATT koristi \textit{Attribute} protokol (ATT) koji ima sličnu funkcionialnost kao SDP\footnote{\textit{Service discovery protocol}, omogućava uređajima da mađusobno saznaju koje servise podržavaju i koje parametre trebaju koristiti pri stvaranju podatkovne veze} protokol kod standardnog Bluetootha, samo što je optimiziran i pojednostavljen za korištenje u uređajima male potrošnje. 
GATT definira na koji način su ATT atributi grupirani tako da čine konkretne servise.
\\

BLE profili su posebne specifikacije koje definiraju servise koji se koriste u određenim scenarijima. 
Od proizvođača konkretnih uređaja se očekuje da zadovolje određen profil kako bi se osigurala međusobna kompatibilnost između raznih uređaja (npr. ako uređaj koji prati krvni tlak zadovolji HRP profil svi uređaji će moći čitati  . 





%TODO BLE značajniji profili, spomenuti da su svi profili temeljeni na gattu
%TODO nešto malo o strukturi paketa, servisima, karakteristikama, opisnicima
%TODO spomenuti koji uređaji imaju podrušku za ble i slično
TODO